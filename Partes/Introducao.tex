\chapter{Introdução}
\label{introducao}

\section{O que é um bot?}
 
    Bot é versão resumida da palavra de língua inglesa robot. Resumidamente, é uma ferramenta automatizada que executa uma série de funções pré-programadas. Normalmente, está associada a inteligência artificial e busca interagir simulando a forma de pensar humana.
 
\subsection{Chatbots}
 
     Essa é a ferramenta também chamada de bot que as pessoas mais terão contato no seu cotidiano. Junção das palavras chat e robot, os chatbots – ou robôs de conversação – são ferramentas de comunicação. Seu propósito é simular uma interação com um agente de atendimento humano e automatizar diversos processos.
 
\section{Por que usar um bot?}

     Se os bots fazem o atendimento do usuário através de canais de chat (no próprio site, no WhatsApp, Telegram, Skype, Facebook Messenger e muitos outros), porque então não continuar com uma pessoa atendendo? A ideia é que o bot complemente o trabalho do atendimento humano e não necessariamente o substitua.
    Uma pessoa conseguiria atender um volume frequente de mensagens, mas demoraria um tempo até responder a todas as solicitações. Além disso, existem muitas questões frequentes repetitivas que tomam muito tempo e acaba limitando a ação para problemas mais complexos. 
    Com um chatbot, sua empresa pode interagir com centenas de usuários simultaneamente. Além oferecer respostas muito mais rápidas, todas essas questões repetitivas podem ser respondidas sem ocupar o tempo de um atendente humano. Além disso, agiliza o processo de atendimento, pois identifica o usuário rapidamente e acessa o histórico de mensagens para continuar do ponto em que parou, sem a necessidade de transferir o atendimento até encontrar o setor responsável.
    Trabalhando em conjunto com a equipe de atendimento, o chatbot pode transferir determinados tópicos mais complexos ou de escolha da empresa para um operador humano. Isso significa que suporte técnico, vendas, preferências do usuário por um atendente, entre outros aspectos, podem ser transferidos para que o time especialista dê continuidade à interação iniciada pelo bot.
