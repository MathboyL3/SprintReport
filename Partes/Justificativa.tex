\section{Problema e Justificativa}
\label{justificativa}

\begin{enumerate}
    \item
        Se os bots fazem o atendimento do usuário através de canais de chat (no próprio site, no WhatsApp, Telegram, Skype, Facebook Messenger e muitos outros), porque então não continuar com uma pessoa atendendo? A ideia é que o bot complemente o trabalho do atendimento humano e não necessariamente o substitua.
        Uma pessoa conseguiria atender um volume frequente de mensagens, mas demoraria um tempo até responder a todas as solicitações. Além disso, existem muitas questões frequentes repetitivas que tomam muito tempo e acaba limitando a ação para problemas mais complexos. 
        Com um chatbot, sua empresa pode interagir com centenas de usuários simultaneamente. Além oferecer respostas muito mais rápidas, todas essas questões repetitivas podem ser respondidas sem ocupar o tempo de um atendente humano. Além disso, agiliza o processo de atendimento, pois identifica o usuário rapidamente e acessa o histórico de mensagens para continuar do ponto em que parou, sem a necessidade de transferir o atendimento até encontrar o setor responsável.
        Trabalhando em conjunto com a equipe de atendimento, o chatbot pode transferir determinados tópicos mais complexos ou de escolha da empresa para um operador humano. Isso significa que suporte técnico, vendas, preferências do usuário por um atendente, entre outros aspectos, podem ser transferidos para que o time especialista dê continuidade à interação iniciada pelo bot.
    
    \item
        Sendo, hoje em dia, muito popular o uso da internet para quase tudo, muitas pessoas vêem um uso malicioso para poderem faturar dinheiro ou vazar dados sigilosos ao público. Com isso em mente, tentaremos resolver este problema diretamente no usuário, auxiliando-o como proceder em certas situações para que não seja enganado.
\end{enumerate}


 

