\documentclass[oneside]{uniaraxatcc} % Para impressão em frente e verso, utilize twoside
\usepackage{graphicx}
\usepackage{mathtext}
\usepackage{listings}
\usepackage{amssymb}
\usepackage{color}
\usepackage{subcaption}
\usepackage[utf8]{inputenc}
\usepackage{multirow}
\usepackage{scalefnt}
% \usepackage{wrapfig}
\usepackage{nomencl}
\usepackage{lipsum}

% \usepackage[citestyle=alphabetic,bibstyle=authortitle,sorting=ynt]{biblatex}

\newcommand{\uniaraxa}{\uppercase{UNIARAXÁ}}
\tipotcc{P} % P:Projeto, M: Monografia
% Nome do Curso
\curso{Sistemas de Informação}

%Qual a habilitação do curso? (Bacharel, Tecnólogo ou Licenciado(a))
\habilitacao{Bacharel}

% Autor(a) do trabalho
\autor{Daniel, Ryan, Náthan e João César}

% Título do Trabalho
\titulo{Artefato Sprint 003}

% Local cidade / estado
\local{Campos Altos}{Minas Gerais}


% Data dia / mês / ano da defesa
\date{14}{Outubro}{2022}

%Define a linha de pesquisa
% - Cultura, Desenvolvimento Humano e Gestão
% - Ambiente, Saúde e Políticas Públicas.
% - Produção, inovação e sustentabilidade.
\linhadepesquisa{Tecnologia, Conhecimento e Inovação}

% Orientador [Tratamento]{Nome}{Gênero: M para Masculino e F para Feminino}
\orientador[Prof. Me]{Humberto Gustavo de Melo}{M}

% Membros da banca, podem ser definidos 2 membros de A a B (sendo que o orientador e o co-orientador já  são membros da banca)

% Primeiro membro da banca (Obrigatório)
%\avaliadorA[Prof. Me.]{Humberto Gustavo de Melo}{M}

% Dedicatória (Opcional)
%\dedicatoria{\input{Partes/Dedicatoria.tex}}

% Agradecimentos (Opcional)
%\agradecimentos{%\noindent\chapter*{\uppercase{Agradecimentos}}

%Agradeçemos à todo o grupo CURUPIRA.}

% Epígrafe (Opcional)
%\epigrafe{\input{Partes/Epigrafe.tex}}

%Resumo (Obrigatório)
\resumo{%\noindent\chapter*{\uppercase{Resumo}}

%Resumo do trabalho do H1 aqui
  
%\vspace{1cm}

%\noindent \textbf{Palavras-chaves}:Jogos eletrônicos educativos, Análise de jogos %educativos, Ensino-aprendizagem, Desenvolvimento de jogos educativos}

% Abstract (Obrigatório)
%\abstract{%\noindent\chapter*{\uppercase{Abstract}}

%  \lipsum[1-3]

%\vspace{1cm}

%\noindent \textbf{Palavras-chaves}:Educational eletronic games, Educational games %analysis, Teaching-learning, Development of educational games}

% Siglas e Abreviações (Opcional)
%\siglas{\input{Partes/Siglas.tex}}


\begin{document}
\pretextual %Indica início dos elementos pré-textuais
\maketitle %Elementos Pré-textuais
\textual %Indica início dos elementos textuais
%\pagestyle{simple}

%\input{Partes/Tema.tex} %Utilizado somente no Projeto de TCC
%
% \lipsum[1-2]
% 
% 
\begin{figure}[h]
\captionsetup[subfigure]{labelformat=empty}
\caption{``Às vezes é preciso dar dois passo para trás para dar um para frente.''}
\includegraphics[width=20cm,height=12cm]{Partes/Imagens/logo.jpg}
\subcaption{Fonte: Criada por João César (2022).}
\end{figure}
% 
%\section{Seção 2}
% 
%  \lipsum[1-9] 
\chapter{Introdução}
\label{introducao}

\section{O que é um bot?}
 
    Bot é versão resumida da palavra de língua inglesa robot. Resumidamente, é uma ferramenta automatizada que executa uma série de funções pré-programadas. Normalmente, está associada a inteligência artificial e busca interagir simulando a forma de pensar humana.
 
\subsection{Chatbots}
 
     Essa é a ferramenta também chamada de bot que as pessoas mais terão contato no seu cotidiano. Junção das palavras chat e robot, os chatbots – ou robôs de conversação – são ferramentas de comunicação. Seu propósito é simular uma interação com um agente de atendimento humano e automatizar diversos processos.
 
\section{Por que usar um bot?}

     Se os bots fazem o atendimento do usuário através de canais de chat (no próprio site, no WhatsApp, Telegram, Skype, Facebook Messenger e muitos outros), porque então não continuar com uma pessoa atendendo? A ideia é que o bot complemente o trabalho do atendimento humano e não necessariamente o substitua.
    Uma pessoa conseguiria atender um volume frequente de mensagens, mas demoraria um tempo até responder a todas as solicitações. Além disso, existem muitas questões frequentes repetitivas que tomam muito tempo e acaba limitando a ação para problemas mais complexos. 
    Com um chatbot, sua empresa pode interagir com centenas de usuários simultaneamente. Além oferecer respostas muito mais rápidas, todas essas questões repetitivas podem ser respondidas sem ocupar o tempo de um atendente humano. Além disso, agiliza o processo de atendimento, pois identifica o usuário rapidamente e acessa o histórico de mensagens para continuar do ponto em que parou, sem a necessidade de transferir o atendimento até encontrar o setor responsável.
    Trabalhando em conjunto com a equipe de atendimento, o chatbot pode transferir determinados tópicos mais complexos ou de escolha da empresa para um operador humano. Isso significa que suporte técnico, vendas, preferências do usuário por um atendente, entre outros aspectos, podem ser transferidos para que o time especialista dê continuidade à interação iniciada pelo bot.

\section{Problema e Justificativa}
\label{justificativa}

\begin{enumerate}
    \item
        Se os bots fazem o atendimento do usuário através de canais de chat (no próprio site, no WhatsApp, Telegram, Skype, Facebook Messenger e muitos outros), porque então não continuar com uma pessoa atendendo? A ideia é que o bot complemente o trabalho do atendimento humano e não necessariamente o substitua.
        Uma pessoa conseguiria atender um volume frequente de mensagens, mas demoraria um tempo até responder a todas as solicitações. Além disso, existem muitas questões frequentes repetitivas que tomam muito tempo e acaba limitando a ação para problemas mais complexos. 
        Com um chatbot, sua empresa pode interagir com centenas de usuários simultaneamente. Além oferecer respostas muito mais rápidas, todas essas questões repetitivas podem ser respondidas sem ocupar o tempo de um atendente humano. Além disso, agiliza o processo de atendimento, pois identifica o usuário rapidamente e acessa o histórico de mensagens para continuar do ponto em que parou, sem a necessidade de transferir o atendimento até encontrar o setor responsável.
        Trabalhando em conjunto com a equipe de atendimento, o chatbot pode transferir determinados tópicos mais complexos ou de escolha da empresa para um operador humano. Isso significa que suporte técnico, vendas, preferências do usuário por um atendente, entre outros aspectos, podem ser transferidos para que o time especialista dê continuidade à interação iniciada pelo bot.
    
    \item
        Sendo, hoje em dia, muito popular o uso da internet para quase tudo, muitas pessoas vêem um uso malicioso para poderem faturar dinheiro ou vazar dados sigilosos ao público. Com isso em mente, tentaremos resolver este problema diretamente no usuário, auxiliando-o como proceder em certas situações para que não seja enganado.
\end{enumerate}


 


\section{Objetivos}
\label{objetivos}

%Assumindo que existe um problema a ser resolvido, apresente qual o objetivo de seu projeto de pesquisa. O que você pretende (ou pretendeu) exatamente fazer. Aqui, deve aparecer a principal ``contribuição'' de seu projeto. Qual é a principal ``coisa'' que você pretende/pretendeu fazer? Qual sua principal entrega? Não é necessário criar uma subseção para cada tipo. Pode haver uma única seção, chamada de ``objetivos'' cujo texto divida-se naturalmente em objetivo geral e objetivos específicos, deixando claro qual caso está sendo tratado em cada momento. Para diferenciar o objetivo geral dos objetivos específicos, siga as seguintes diretrizes:

\begin{itemize}
		\item \textbf{Objetivo geral}: 
		    Tivemos como objetivo geral construir um bot de inteligência artificial utilizando a plataforma de escolha que preferir, podendo até mesmo criar o seu bot do zero, não foi definido um tema em especifico para o bot ou seja nos os desenvolvedores também tínhamos como objetivo escolher um tema para o bot, desenvolvemos esse bot para que as pessoas possam obter dicas em geral para o assunto escolhido.
		
		\item \textbf{Objetivos específicos}: 
    		\begin{enumerate}
    		    \item
    		        Desenvolver e implementar um chatbot capaz de conversar com um usuário;
    		    \item 
    		        Pesquisar sobre como construir e funcionamento de Bots;
    		    \item
    		        Escolher uma plataforma para desenvolvimento;
    		    \item 
    		        Identificar o público-alvo do projeto;
    		    \item 
    		        Colocar as tecnologias utilizadas e os motivos das escolhas;
    		    \item 
    		        Criar um plano de testes;
    		    \item 
    		        Decidir como será disponibilizado a versão final do projeto;
    		    \item 
    		        Medir a eficiência de acerto do bot;
    		    \item 
    		        Demonstrar e programar o bot para os casos que não souber de algo perguntando;
    		\end{enumerate}
    	
\end{itemize}

%\input{Partes/Hipotese.tex}
%\input{Partes/Limitacoes.tex}
%\input{Partes/Estrutura.tex}
%\input{Partes/Metodologia.tex}
\chapter{Tecnologias Utilizadas}
    Optamos por não usar uma tecnologia pronta, e decidimos criar nosso chat bot do zero usando C\# (\citeonline{csharp}), assim, possuimos total controle do que queremos que ele faça ou não.
    \par Podemos mudar textos, implemntá-lo em vários lugares diferentes, utilizar tecnologias mais avançadas para melhorar a interação dele com o usuário.
    \par Qualquer problema que aparecer podemos consetá-lo, melhorar o bot sempre que possível, sem depender de terceiros.
\chapter{\textbf{Desenvolvimento}}
\section{Construção do Projeto}

\subsection{Como o BOT Funciona (Simplificado)}
    O Bot Curupira funciona, simplificadamente, da seguinte maneira:
    (\citeonline{FFNN})
    \begin{enumerate}
        \item Um usuário manda alguma mensagem para o número do WhatsApp que o bot usa.
        
        \item O bot, usando a API \textit{Selenium WebDriver}, seleciona o contato que possui mensagens não lidas.
        
        \item Em seguida, o bot procura pela última mensagem enviada no chat e verifica se ela veio do usuário, ou seja, se a mensagem não é do próprio bot.
        
        \item Então, por fim, o bot retira as pontuações da frase, substitui as palavras por pseudônimos definidos no arquivo data.txt, remove palavras que não dão sentidos para a frase e processa a frase em vetores de \textit{Word Embedding}, após isso, ele cria uma vetor único que é a média aritmetica de todos os vetores das palavras reconhecidas.
        
        \item Após isso, esse vetor é passado por uma rede neural treinada que vai liberar como \textit{output} (saída) um vetor com o mesmo tamnho que a quantidade de respostas definidas pelo programador.
        
        \item Finalmente, o index do maior valor desse vetor é o index da resposta que o bot considera como ''certa''.
        
        \item A resposta escolhida pelo bot é enviada ao usuário usando, novamnete, o \textit{Selenium WebDriver}.
    \end{enumerate}

\subsection{Como usamos o WebDriver}
    O Google Driver foi essencial para a comunicação pelo \textit{WhatsApp}. No mercado, quase todas as API que fornecem acesso às messagens e interações com \textit{WhatsApp} são pagas ou possuem muitas limitações, então, foi necessário chegarmos numa solução própria. A nossa solução consiste no uso de um WebDriver - são versões do navegador que permitem o controle externo das funções dele, pricipalmente, do código-fonte da página web que está aberta. Com isso, conseguimos acessar o HTML da página do \textit{WhatsApp Web} e clicar e mudar textos de elementos da página.
    
    
    Assim, conseguimos usar CssSelector para encontrar elementos específicos, por exemplo, no Bot Curupira, procuramos por todas as mensagens que estão na tela após selecionar um contato com mensagens não lidas, depois verificamos se a última mensagem não é do próprio bot e respondemos ela se não for do bot. Para responder, pegamos o texto do elemento da última mensagem, passamos para um script em C\# que consiste de uma rede neural treinada que recebe vetores de 100 floats.
    \citeonline{SleniumWebDriver}
\section{Base de dados Q\&A}
    Não usamos um modelo de Questão - Resposta para a AI. Com o poder de Machine Learning podemos processar as perguntas em vetores e passarmos por uma Rede Neural (FFNN) para chegar na resposta com mais chance de ser certa. Usamos um modelo de Questão - Resposta apenas para validar, artificialmente, a taxa de acerto do bot.
    
\subsection{Base de Dados de Treinamento e Teste}
    Todas nossas bases de dados são guardadas localmente em arquivos de texto (.txt). O arquivo onde guarda-se as informações para treino, respostas e pseudônimos deve seguir a seguinte formatação:
    Baseando-se no uso de ''blocos'' para separar cada informação, devem ser usadas as seguintes tags para começar e terminar blocos:\newline
    \textbf{[item]} <- início de um bloco\newline
    \textbf{[*item]} <- final de um bloco\newline
    \par Para respostas deve-se usar apenas um bloco:\newline
    \textbf{[Respostas]}\newline
    ...\newline
    \textbf{[*Respostas]}
    \par
    Para facilitar o uso desse ambiente, pode-se usar o símbolo ''|'' para separar a resposta da numeração dela, a enumeração começa em 1, durante a leitura do arquivo, tudo antes e incluindo o símbolo ''|'' serão ignorados e o número da resposta a partir do início do bloco que será usado para referenciar ela, basta começar enumerando (1, 2, 3, 4...):\newline
    \textbf{[Respostas]}\newline
    1|Resposta 1...\newline
    2|Resposta 2...\newline
    3|Resposta 3...\newline
    \textbf{[*Respostas]}
    \par
    Para os pseudônimos usa-se a tag ''alias'', os pseudônimos são definidos a partir de uma palavra principal(\#) e outras palavras(\&) que serão substitídas pela palavra principal:\newline
    \textbf{[Alias]}\newline
    \#antivírus\newline
    \&anti-vírus\newline
    \textbf{[*Alias]}\newline
    \par
    Ainda não bem definido e não está em uso é a função do saiba mais. Ela servirá para enviar uma mensagem com um link falando melhor sobre tema abordado na última resposta do bot. Ela será usada quando a resposta do bot não for sufuciente para tirar a dúvida do usuário, ou seja, quando o usuário não entender e falar coisas como: ''não entendi...'', ''como assim?'', ''mais informações'' e etc. 
    Será passado ao usuário, se existir, o saiba mais definido. A implementação dele é semelhante à implementação das respostas:\newline
    \textbf{[saiba\_mais]}\newline
    2|Link\_2...\newline
    1|Link\_1...\newline
    3|Link\_3...\newline
    \textbf{[*saiba\_mais]}
    \par
    Agora, partindo para a implementação de dados para treino e verificação de eficiencia.\par
    Para a implentação de dados de treino usamos a tag ''dados'', nela usamos os símbolos de ''\texttt{@}'' para as palavras chaves, ou seja, palavras que, quando encontradas na frase, trarão o significado daquel tema, lembrando de usar as palavras principais definidas nos pseudônimos para melhor taxa de acerto, você também pode usar o ''\texttt{;}'' para adicionar mais palavras chaves, que serão convertidas todas em um único vetor. O símbolo ''\texttt{=}'' é usado para determinar qual a resposta certa para aquelas palavras chaves definidas, só deve haver um por bloco de dados.\par
    Ainda não está implementado mas com o uso do sinal ''!'' é possível definir uma opção de saiba mais para o bloco de dados.\newline
    \textbf{[dados]}\newline
    @firewal;significado\newline
    =1\newline
    !2 (opcional)\newline
    \textbf{[*dados]}\par
    Por fim, temos o bloco de teste, é como ele que avaliamos, artificialmente, a eficiência do bot. Para isso definimos apenas UM bloco, dentro dele usamos pa ''@'' definirmos perguntas e ''='' para definirmos a resposta certa. Exemplo:\newline
    \textbf{[teste]}\newline
    @o que é firewall?\newline
    @como funciona o firewall?\newline
    =4\newline
    'espaço em branco\newline
    @o que é um anti virus?\newline
    =1\newline
    \textbf{[*teste]}\par
    Podemos ver que no exemplo acima, foi usado o símbolo de aspas simples, come ele, você pode definir comentários dentro do arquivo, esse comentários serão ignorados durante a leitura do arquivo.
    
    
    
    
    
    

\chapter{Plano de Testes}
\section{Teste Artificial}
    Os testes artificiais são usados para determinar a usabilidade da rede neural. O motivo de usarmos um teste artificial primeiramente é para que consigamos selecionar um bot com uma maior capacidade de acerto. Durante os treinos a as taxas de acertos variam de 0.1\% até 10\% normalmente, porém, em algumas raras instâncias, esssa mudança pode ser de até de 40\% negativo, ou seja, de um treinamento para outro a eficiência do bot pode cair 40\%, e se não treinarmos ele novamnete, ele não irá satizfazer os póximos testes.
\section{Teste Entre Grupos}
    Esse será o segundo teste que será executado, não será o último, pois nosso chat bot não é direcionado principalmente ao público que já possui algum conhecimento sobre o assunto e sim aqueles que necessitam desse conhecimento. Claro que todos poderão usar o bot. Esse teste seria mais um lançamneto BETA para testes, onde, provavelmente aparecerão novas perguntas, novos temas sobre o assunto de segurança digital e outros, que serão adicionados à base dados para treinamentos de novas redes neurais.
\section{Teste com Pessoas Externas}
    Esse será o teste definitivo, onde não sabemos o nível de conhecimento de quem está conversando com o bot, ou seja, pode ser uma pessoa técnica que perguntará coisas avançadas ou pessoas leigas que perguntará coisas muito básicas. Em ambos os casos, podem aparecer temas não pensados e esse é o objetivo dos testes, encontrar fraquezas para serem fortalecidas.
\section{Acesso}
    O acesso ao BOT curupira poderá ser feito via \textit{WhatsApp}, pelo número:\newline \textbf{\texttt{+}55 37 3426-1485}.\par
    
    Durante os acessos em grande escala, conseguiremos testar a estabilidade do nosso bot (tivemos problemas de estabilidade da última vez, onde o bot não conseguiu lidar com tantas pessoas perguntando ao mesmo tempo, esse problema pode ter sido minimizado na nova versão de nossa biblioteca C\# WhatsLib) e a taxa de acerto dele referente às perguntas feitas pelos usuários.
\section{Plano de Ação}
    Para conseguirmos alcançar um número considerável de pessoas diferentes, divuldaremos o nosso bot por grupode WhatsApp (grupos de família e amigos), gupos do discord e outros.\par
    Acreditamos que, assim, conseguiremos ser imparcial na escolha das pessoas externas e conseguiremos um gradiente bem disperso com pessoas com diferentes níveis de conhecimento.
    
\chapter{\uppercase{Trabalhos Relacionados}}
\label{TrabalhosRelacionados}

\section{Atendente Virtual Grazziotin}
    
    \begin{center}
        “Um sistema chatbot para atendimentos aos usuários da empresa Grazziotin” (\cite{Grazziotin})
    \end{center}
    
    \subsection{Objetivo}
        Desenvolver uma ferramenta de atendimento, de forma automática, assim orientando os usuários/colaboradores com o uso desta ferramenta, seus acessos, cadastros e manutenções do sistema, para assim, reduzir o excesso de atendimentos e ligações realizadas ao setor de TI.
        
    \subsection{Relação com o nosso trabalho}
        O sistema de Chatbots, que é um sistema de conversas avançado que foi criado para interações entre máquina e usuários e possuem algoritmos de aprendizado avançado, trabalham com atendimento imparcial, já que é possível criar as ações e respostas, bem como atender vários usuários ao mesmo tempo e executar determinadas ações no sistema.
    
    \subsection{Diferenças}
        \begin{enumerate}
            \item 
                A ferramenta que eles desenvolveram tem como principal objetivo solucionar problemas rotineiros que ocorrem no dia a dia de cada empresa, enquanto nosso chatbot está ligado a solucionar problemas de segurança digital.
            \item
                O bot de atendimento virtual Grazziotin tem como linguagens principais: Javascript e CSS e o nosso C\#.
        \end{enumerate}
    
    \subsection{Semelhanças}
        \begin{enumerate}
            \item[CUI] 
                As interfaces de conversação emulam conversações com o usuário. As CUIs permitem que o usuário se comunique com o computador utilizando linguagem natural. Para fazer isso, as interfaces de conversação usam o processamento de linguagem natural (PNL) para permitir que os sistemas computacionais entendam, analisem e criem significado a partir da linguagem humana. O PNL considera a estrutura da linguagem humana buscando compreender as intenções que o usuário está tentando assinalar.
            
        \end{enumerate}

\section{Ágata}
    \begin{center}
        "ÁGATA: um chatbot para difusão de práticas para Educação Ambiental"(\cite{Agata})
    \end{center}

    \subsection{Objetivo}
        Utilizar um algoritmo chatbot em python no telegrama para promover conhecimento geral sobre Educação Ambiental, como: dicas e curiosidades para termos hábitos mais sustentáveis.
    
    \subsection{Relação com o nosso trabalho}
        Os 2 trabalhos apresentam sistemas de chatbot com o intuito de espalhar conhecimento, tendo como principal função a segurança e disponibilidade de informações, esses sistemas de chatbots servem para amenizar os problemas que ocorrem com bastante frequência e que podem ser resolvidos com informações pré escritas, diminuindo assim a frequência que esses problemas ocorrem e agilizando a forma que possam ser resolvidos.
    
    \subsection{Diferenças}
        \begin{enumerate}
            \item 
                A linguagem de programação princiapl utilizada pelo bot Ágata é python e o banco de dados deles é em MySQL.
            \item 
                Objetivos diferentes, enquanto o bot Ágata é usado para espalhar fatos sobre Educação Ambiental o nosso tira dúvidas sobre segurança digital.
        \end{enumerate}
    
    \subsection{Semelhanças}
        A arquitetura do chatbot é baseada no modelo cliente-servidor, o nosso também é baseado nesse meodel, onde o usuário usa a plataforma WhatsApp para transferir a mensagem para o nosso servidor e o servidor envia devolta ao usuário pelo WhatsApp. 
\chapter{\uppercase{Finalização}}
\label{conclusao}

Esperamos que com essa contribuição, a sociedade ao redor da internet se torne cada vez mais segura e que essas informações sobre a segurança se torne cada vez mais acessível.

\section{Disponibilização}
    A disponibilização da interação com o chat bot será feita via \textit{WhatsApp} (\citeonline{whatsapp}) e a do projeto será feita pelo GitHub (\citeonline{github}).\par
    O objetivo de estarmos disponibilizando esse projeto como Open Source é para encorajar outras pessoas a se ingressar no meio da programação, visto que há várias possibilidades para o uso do conhecimento.]
    
\section{O que Oferecemos?}
    O BOT Curupira um simples bot com interação com o \textit{WhatsApp}, tornando-o accesível por diversas pessoas.\par
    Do que jeito que está configurado por padrão, ele consegue responder diversas perguntas sobre segurança digital.


 \section{Planos Futuros}
    \begin{itemize}
        \item Planejamos refazer nossa biblioteca de Matrizes e nossa Rede Neural para que possa diminuir o tempo de treino dela, apesar de já estar num nível considerável.
    \end{itemize}

%\input{Partes/Etapas.tex} %Utilizado somente no Projeto de TCC
%\input{Partes/Cronograma.tex} %Utilizado somente no Projeto de TCC


% \postextual %Indica início dos elementos pós-textuais	

\bibliography{Referencias}


%\begin{anexosenv}
% Imprime uma página indicando o início dos anexos
	%\partanexos

%(opcional)
%\input{Partes/Anexo.tex} 
%\end{anexosenv}


%(opcional)
%\chapter{\uppercase{Considerações finais}}
\label{conclusao}

%Dicas para desenvolver a estrutura das considerações finais:

%\begin{itemize}

%\item Contextualize o projeto desenvolvido, ou seja, ajude o leitor a "relembrar" a estrutura. 

%\item Apresente os resultados.

%\item Apresente as contribuições do projeto para o meio acadêmico e/ou para o mercado. 

%\item Se for possível, descreva sobre melhorias para próximas pesquisas.

%\end{itemize}

%Neste Capítulo pode ser inserido as Seções: Trabalhos Futuros, Contribuições do Projeto, Artigos publicados. 

 Antes de iniciar este Trabalho de Conclusão de Curso, já havia diversos relatos falados vindos de conversas e palestras anteriores de como jogos eletrônicos eram importantes para outras atividades além de divertir, inclusive a pedagogia, na qual já se sabia que este conceito era experimentado faz alguns anos por conta da existência de jogos educativos. Ao desenvolver a pesquisa feita aqui, o objetivo inicial era confirmar estes relatos, usando a área de educação como exemplo, além de desenvolver o jogo educativo em paralelo mencionado anteriormente para treinar o desenvolvimento futuro de jogos e fazer um jogo educativo de matemática que pode possivelmente ser distribuído gratuitamente em escolas públicas.
 
 O desenvolvimento do jogo educativo ``Number Invasion" demorou, mas foi uma experiência enriquecedora. Foi um prazer trabalhar nele por conta da pretensão de seguir a área de desenvolvimento de jogos eletrônicos no futuro e adquirir experiência em um projeto como este era fundamental. Mais conhecimento sobre o Unity e a linguagem de programação C\# foi adquirido conforme o projeto foi se realizando, e diversas dúvidas sobre como alguns algoritmos eram feitos foram respondidas graças a enorme comunidade do Unity localizadas em fóruns e vídeos pela Internet, e ``Number Invasion" então se tornou o primeiro jogo completo feito, o que deu experiência valiosa e abre as portas para projetos futuros sem cunho educacional.
 
 A pesquisa foi a parte mais difícil deste Trabalho de Conclusão de Curso, pois explorava um campo novo onde não se havia experiência prévia como foi o desenvolvimento do jogo. Conforme a pesquisa se desenvolveu, o assunto foi aprendido aos poucos e o material da pesquisa foi complementando o jogo educativo e, com os dois unidos ao final da pesquisa, um completava o outro, onde a pesquisa mostrava a parte teórica e o jogo a parte prática.
 
 Com tudo o que foi apresentado ao decorrer deste Trabalho de Conclusão de Curso, conclui-se que os objetivos foram alcançados. O jogo educativo feito em paralelo, apesar de ter destino incerto quanto a ser distribuído em escolas, foi finalizado com sucesso sem apresentar erros e foi um bom treino para projetos futuros, e todas as evidências apresentadas de como o cérebro é afetado por jogos eletrônicos mostram que os mesmos têm grande potencial de serem usados como ferramentas auxiliares em diversas áreas, inclusive a pedagogia. Porém, é necessário estudar antes como será a implementação dos jogos eletrônicos em sala de aula, para que os mesmos sejam eficientes em sua função de ensinar e o máximo de conhecimento seja extraído.
 
  \section{Trabalhos Futuros}
  
  Este jogo não foi testado em sala de aula, portanto, sua capacidade de ensinar e ser divertido é desconhecida. Serão necessários testes práticos no futuro para comprovar sua eficácia. 

\end{document}
